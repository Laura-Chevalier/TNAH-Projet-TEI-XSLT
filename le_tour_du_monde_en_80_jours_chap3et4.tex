\documentclass[]{book}
            \usepackage[utf8]{inputenc}
            \usepackage[T1]{fontenc}
            \usepackage[a4paper]{geometry}
            \geometry{left=3cm, right=3cm, top=3cm, bottom=3cm}
            \usepackage[french]{babel}
            \usepackage[protrusion=true,expansion=true,letterspace=50,wordspacing=true]{microtype}
            \usepackage{fancyhdr}
            \usepackage{titlesec}
            \titleformat{\chapter}[block]{\Large\bfseries}{}{1em}{}
            
            \usepackage[splitindex]{imakeidx}
            \makeindex[name=persName,title=Index des noms de personnages]
            \makeindex[name=placeName,title=Index des noms de lieux]
            
            \usepackage[hidelinks, pdfstartview=FitH, plainpages=false]{hyperref} 
 
            \title{\textbf{Le Tour du monde en quatre-vingts jours}}
            \author{
               Jules 
               Verne
            }
            \date{7 novembre 1872}
            
            \begin{document}
            \maketitle
  
            \pagestyle{fancy}
            \fancyhf{}
            \fancyfoot[C]{\thepage}
            \renewcommand{\headrulewidth}{1pt}

    
 Ce document PDF a été compilé à partir d'une feuille de style XSL rédigée dans le cadre du Master TNAH de l'École nationale des chartes pour le cours de Jean-Damien GENERO.
            Il contient les chapitres III et IV du \textit{Le Tour du monde en quatre-vingts jours} de 
               Jules 
               Verne
            , publié alors en roman-feuilleton dans \textit{Le Temps}. Ces chapitres ont été publiés dans les colonnes du journal le 7 novembre 1872.
        
\newpage
    
            
               
         
               
         
            \chapter{Chapitre III}
            
            
                  
               Phileas Fogg\index[persName]{{Phileas Fogg}} avait quitté sa maison de
               Saville-row\index[placeName]{{Saville-Row}} à onze heures et demie; et 
               après avoir placé treize cent soixantequinze fois son pied droit devant son pied
               gauche et treize cent soixante-seize fois
               son pied gauche devant son pied droit, il
               arriva au Reform-Club\index[placeName]{{Reform-Club}}, vaste édifice élevé
               dans Pall-Mall\index[placeName]{{}}, et qui n'a pas coûté moins
               de trois millions à bâtir.
               
               Phileas Fogg\index[persName]{{Phileas Fogg}} se rendit aussitôt à la
               salle à manger, dont les neuf fenêtres
               s'ouvraient sur un beau jardin aux arbres
               déjà dorés par l'automne. Là, il prit place
               à la table habituelle où son couvert l'attendait; son déjeuner se composait d'un
               hors-d’oeuvre, d'un poisson bouilli relevé
               d'une reading sauce de premier choix,
               d'un rosbeaf  écarlate agrémenté de con
               diments musheron, d'un gâteau farci
               de tiges de rhubarbe et de groseilles ver
               \footnote{Voir Le Temps du 6 novembre.}  
               
               tes, d'un morceau de chester, le tout arrosé de quelques tasses d'un thé spécialement recueilli pour l'office du ReformClub\index[placeName]{{Reform-Club}}.
               
               A midi quarante-sept, ce gentleman se
               leva et se dirigea vers le grand salon,
               somptueux hall orné de peintures richement encadrées. Là, un domestique lui
               remit le Times non coupé, dont Phileas 
               Fogg\index[persName]{{Phileas Fogg}} opéra le laborieux dépliage avec une 
               sûreté de main qui dénotait une grande
               habitude de cette difficile opération. La
               lecture de ce journal occupa Phileas Fogg\index[persName]{{Phileas Fogg}}
               jusqu'à trois heures quarante-cinq, et celle
               du Daily Telegraph, - qui lui succéda,dura jusqu'au dîner. Ce repas s'accomplit
               dans les mêmes conditions que le déjeuner avec adjonction de royal british 
               sauce.
            
               A huit heures moins vingt, le gentleman
               reparut dans le grand salon et s'absorba
               dans la lecture du Morning Chronicle.
            
               Une demi-heure plus tard, divers habitués du Reform-Club\index[placeName]{{Reform-Club}} faisaient leur entrée et s'approchaient de la cheminée où 
               brûlait un feu de houille. C'étaient les
               partenaires habituels de Mr. Phileas Fogg\index[persName]{{Phileas Fogg}},
               comme lui, enragés joueurs de whist :
               l'ingénieur Andrew Stuart\index[persName]{{Andrew Stuart}}, les banquiers 
               John Sullivan\index[persName]{{John Sullivan}} et Samuel Fallentin\index[persName]{{Samuel Fallentin}} le 
               brasseur Thomas Flanagan\index[persName]{{Thomas Flanagan}}, Gauthier 
               Ralph\index[persName]{{Gauthier Ralph}}, un des administrateurs de la Banque d'Angleterre\index[placeName]{{Banque d'Angleterre}}, personnages riches et
               considérés, même dans ce club qui compte
               parmi ses membres les sommités de l'industrie et de la finance.
            
               « Eh bien! Ralph\index[persName]{{Gauthier Ralph}}, demanda Thomas
               Flanagan\index[persName]{{Thomas Flanagan}}, où en est cette affaire de vol? \newline
               -Eh bien, répondit Andrew Stuart\index[persName]{{Andrew Stuart}}, la
               Banque\index[placeName]{{Banque d'Angleterre}} en sera pour son argent.\newline
               -J'espère, au contraire, dit Gauthier 
               Ralph\index[persName]{{Gauthier Ralph}}, que nous mettrons la main sur  
               l'auteur du vol. Des inspecteurs de police, 
               gens fort habiles, ont été envoyés en Amé- 
               rique\index[placeName]{{Amérique}} et en Europe\index[placeName]{{Europe}}, dans tous les principaux ports d'embarquement et de débarquement, et il sera difficile à ce monsieur 
                de leur échapper.\newline
               -Mais on a donc le signalement du vo
               
                  
               leur? demanda Andrew Stuart\index[persName]{{Andrew Stuart}}.\newline
               -D'abord, ce n'est pas un voleur, répondit sérieusement Gauthier Ralph\index[persName]{{Gauthier Ralph}}.\newline
               -Comment? ce n'est pas un voleur, cet
               individu qui a soustrait pour cinquante
               cinq mille livres de bank-notes (1 million
               375,000 francs) ?\newline
               -Non, répondit Ralph\index[persName]{{Gauthier Ralph}}.\newline
               -C'est donc un industriel? dit John
               Sullivan\index[persName]{{John Sullivan}}.\newline
               -Le Morning Chronicle assure que c'est 
               un gentleman. »\newline
            
               Celui qui fit cette réponse n'était autre 
               que Phileas Fogg\index[persName]{{Phileas Fogg}}, dont la tête émergeait
               alors du flot de papier amassé autour de
               lui. En même temps, Phileas Fogg\index[persName]{{Phileas Fogg}} salua ses collègues, qui lui rendirent son 
               salut.
            
               Le fait dont il était question, que les divers journaux du Royaume-Uni\index[placeName]{{Royaume-Uni}} discutaient avec ardeur, s'était accompli trois
               jours auparavant, le 29 septembre. Une 
               liasse de bank-notes, formant l'énorme 
               somme de cinquante-cinq mille livres,
               avait été prise sur la tablette du caissier
               principal de la Banque d'Angleterre\index[placeName]{{Banque d'Angleterre}}.
            
               A qui s'étonnait qu'un tel vol eût pu
               s'accomplir aussi facilement, le sous-gouverneur Gauthier Ralph\index[persName]{{Gauthier Ralph}} se bornait à répondre qu'à ce moment même, le caissier 
               s'occupait d'enregistrer une recette de 
               trois shillings six pences et qu'on ne saurait
               avoir l'œil à tout.
            
               Mais il convient de faire observer ici ce qui rend le fait plus explicable - que
               cet admirable établissement de Bank of 
               England\index[placeName]{{Banque d'Angleterre}} parait se soucier extrêmement 
               de la dignité du public. Point de gardes,
               point d'invalides, point de grillages ! L'or,
               l'argent, les billets sont exposés librement
               et pour ainsi dire à la merci du premier
               venu. On ne saurait mettre en suspicion
               l'honorabilité d'un passant quelconque. Un
               des meilleurs observateurs des usages anglais raconte même ceci : Dans une des 
               salles de la Banque\index[placeName]{{Banque d'Angleterre}} où il se trouvait un
               jour, il eut la curiosité de voir de plus
               près un lingot d'or pesant sept à huit livres, qui se trouvait exposé sur la tablette 
               
               du caissier; il prit ce lingot, l'examina, le
               passa à son voisin, celui-ci à un autre, si 
               bien que le lingot, de main en main, s'en 
               alla jusqu'au fond d'un corridor obscur, et 
               ne revint, qu'une demi-heure après, reprendre sa place, sans que le caissier eût seulement levé la tête.
            
               Mais, le 29 septembre, les choses ne se 
               passèrent pas tout à fait ainsi; la liasse
               de bank-notes ne revint pas et quand la
               magnifique horloge, posée au-dessus du 
               drawing-office sonna à cinq heures la
               fermeture des bureaux, la Banque d'Angleterre\index[placeName]{{Banque d'Angleterre}} n'avait plus qu'à passer cinquantecinq mille livres par le compte de profits 
               et pertes.
            
               Le vol bien et dûment reconnu, des
               agents, des « détectives » choisis parmi les
               plus habiles, furent envoyés dans les principaux ports, à Liverpool\index[placeName]{{Liverpool}}, à Glasgow\index[placeName]{{Glasgow}}, au
               Havre\index[placeName]{{Le Havre}}, à Suez\index[placeName]{{Suez}}, à Brindisi\index[placeName]{{Brindisi}}, à NewYork\index[placeName]{{New-York}}, etc., avec promesse, en cas de succès, d'une prime de deux mille livres
               (50,000 fr.) et cinq pour cent de la somme
               qui serait retrouvée. En attendant les 
               renseignements que devait fournir l'enquête immédiatement commencée, ces inspecteurs avaient pour mission d'observer
               scrupuleusement tous les voyageurs en 
               arrivée ou en partance.
            
               Or, précisément, ainsi que le disait le 
               Morning Chronicle, on avait lieu de supposer que l'auteur du vol ne faisait partie
               d'aucune des sociétés de voleurs d'Angleterre\index[placeName]{{Royaume-Uni}}. Pendant cette journée du 29 septembre, un gentleman bien mis, de bonnes 
               manières, l'air distingué, avait été remarqué, qui allait et venait dans la salle des 
               payements, théâtre du vol. L'enquête avait
               permis de refaire assez exactement le signalement de ce gentleman, signalement
               qui fut aussitôt adressé à tous les détectives du Royaume-Uni\index[placeName]{{Royaume-Uni}} et du continent. 
               Quelques bons esprits - et Gauthier Ralph\index[persName]{{Gauthier Ralph}}
               était du nombre - se croyaient donc fondés à espérer que le voleur n'échapperait
               pas.
            
               Comme on le pense, ce fait était à l'ordre du jour à Londres\index[placeName]{{Londres}} et dans toute l' 
                  
               Angleterre\index[placeName]{{Royaume-Uni}}.On discutait, on se passionnait
               pour ou contre les probabilités du succès 
               de la police métropolitaine. On ne s'étonnera donc pas d'entendre les membres du 
               Reform-Club\index[placeName]{{Reform-Club}} traiter la même question, 
               d'autant plus que l'un des sous-gouver
               neurs de la Banque\index[placeName]{{Banque d'Angleterre}} se trouvait parmi
               eux.
            
               L'honorable Gauthier Ralph\index[persName]{{Gauthier Ralph}} ne voulait 
               pas douter du résultat des recherches, estimant que la prime offerte devrait singulièrement aiguiser le zèle et l'intelligence
               des agents. Mais son collègue, Andrew 
               Stuart\index[persName]{{Andrew Stuart}}, était loin de partager cette confiance. La discussion continua donc entre
               les gentlemen qui s'étaient assis à la table
               de whist, Stuart\index[persName]{{Andrew Stuart}} devant Flanagan\index[persName]{{Thomas Flanagan}}, Fallentin\index[persName]{{Samuel Fallentin}} devant Phileas Fogg\index[persName]{{Phileas Fogg}}. Pendant le jeu,
               les joueurs ne parlaient pas, mais entre
               les robbres, la conversation interrompue
               reprenait de plus belle.
            
               « Je soutiens, dit Andrew Stuart\index[persName]{{Andrew Stuart}}, que
               les chances sont en faveur du voleur qui
               ne peut manquer d'être un habile homme !\newline
               -Allons donc! répondit Ralph\index[persName]{{Gauthier Ralph}}, il n'y a
               plus un pays dans lequel il puisse se réfugier.\newline
               -Par exemple. \newline
               -Où voulez-vous qu'il aille? \newline
               -Je n'en sais rien, répondit Andrew 
               Stuart\index[persName]{{Andrew Stuart}}; mais, après tout, la terre est assez vaste.\newline
               -Elle l'était autrefois… » dit à mi-voix 
               Phileas Fogg\index[persName]{{Phileas Fogg}}; puis « à vous de couper, 
               monsieur », ajouta-t-il en présentant les
               cartes à Thomas Flanagan\index[persName]{{Thomas Flanagan}}.\newline
            
               La discussion fut suspendue pendant le
               robbre. Mais bientôt, Andrew Stuart\index[persName]{{Andrew Stuart}} la reprenait, disant :
            
               « Comment, autrefois! Est-ce que la 
               terre a diminué, par hasard?\newline
               -Sans doute, répondit Gauthier Ralph\index[persName]{{Gauthier Ralph}};
               je suis de l'avis de Mr. Fogg\index[persName]{{Phileas Fogg}}. La terre a
               diminué puisqu'on la parcourt maintenant 
               dix fois plus vite qu'il y a cent ans. Et
               c'est ce qui, dans le cas dont nous nous 
               occupons, rendra les recherches plus rapides.\newline 
               
               -Et rendra plus facile aussi la fuite du 
               voleur !\newline
               -A vous de jouer, monsieur Stuart\index[persName]{{Andrew Stuart}} ! »
               dit Phileas Fogg\index[persName]{{Phileas Fogg}}.\newline
               Mais l’incrédule Stuart\index[persName]{{Andrew Stuart}} n'était pas convaincu, et la partie achevée :
               « Il faut avouer, monsieur Ralph\index[persName]{{Gauthier Ralph}}, repritil, que vous avez trouvé là une manière 
               plaisante de dire que la terre a diminué!
               Ainsi parce qu'on en fait maintenant le 
               tour en trois mois.\newline
               -En quatre-vingts jours seulement, dit 
               Phileas Fogg\index[persName]{{Phileas Fogg}}.\newline
               -En effet, messieurs, ajouta John Sullivan\index[persName]{{John Sullivan}}, quatre-vingts jours depuis que la
               section entre Rothal\index[placeName]{{Rothal}} et Allahabad\index[placeName]{{Allahabad}} a été 
               ouverte sur le « Great-Indian peninsular
                railway » et voici le calcul établi par le
                Morning Chronicle :
                  
               De Londres\index[placeName]{{Londres}} à Suez\index[placeName]{{Suez}} par le mont Cenis\index[placeName]{{Mont Cenis}} et
               Brindisi\index[placeName]{{Brindisi}}, railways et paquebots. 7 jours.
               De Suez\index[placeName]{{Suez}} à Bombay\index[placeName]{{Bombay}}, paquebot. 13 
               De Bombay\index[placeName]{{Bombay}} à Calcutta\index[placeName]{{}}, railway. 3 
               De Calcutta\index[placeName]{{Calcutta}} à Hong-Kong\index[placeName]{{}} (Chine), paquebot. 12 
               De Hong-Kong\index[placeName]{{}} à Yokohama\index[placeName]{{Yokohama}} (Japon), paquebot. 6
               De Yokohama\index[placeName]{{Yokohama}} à San Francisco\index[placeName]{{San Francisco}}, paquebot. 22 
               De San Francisco\index[placeName]{{San Francisco}} à New- York\index[placeName]{{New-York}},railway. 7 
               De New-York\index[placeName]{{New-York}} à Londres\index[placeName]{{Londres}}, paquebot et railway. 10 
               Total. 80 jours.\newline
            
               -Oui, quatre-vingts jours, s'écria Andrew Stuart\index[persName]{{Andrew Stuart}}, qui, par inattention, coupa 
               une carte maîtresse, mais non compris le 
               mauvais temps, les vents contraires les
               naufrages, les déraillements, etc.\newline
               -Tout compris, répondit Phileas Fogg\index[persName]{{Phileas Fogg}},
               en continuant de jouer, car, cette fois, la 
               discussion ne respectait plus le whist.\newline
               -Même si les Indous ou les Indiens enlèvent les rails! s’écria Andrew Stuart\index[persName]{{Andrew Stuart}},
                  
               
                  
               
                  
               s'ils arrêtent les trains, pillent les fourgons et scalpent les voyageurs !\newline
               -Tout compris, » répondit Phileas Fogg\index[persName]{{Phileas Fogg}}\newline,
               qui abattit son jeu, deux atouts maîtres.
               Andrew Stuart\index[persName]{{Andrew Stuart}} a qui c’était le tour de
                « faire », ramassa les cartes en disant:
               « Théoriquement, vous avez raison,
               monsieur Fogg\index[persName]{{Phileas Fogg}}, mais dans la pratique…\newline
               -Dans la pratique aussi, monsieur 
               Stuart\index[persName]{{Andrew Stuart}}.\newline
               -Je voudrais bien vous y voir.\newline
               -Il ne tient qu'à vous. Partons ensemble.\newline
               -Le ciel m'en préserve ! s'écria Stuart\index[persName]{{Andrew Stuart}},
               mais je parierais bien quatre milles livres 
               (100,000 fr.) qu'un tel voyage, fait dans ces
               conditions, est impossible.\newline
               -Très possible, au contraire, répondit
               Mr.Fogg\index[persName]{{Phileas Fogg}}.\newline
               -Eh bien, faites-le donc!\newline
               -Le tour du monde en quatre-vingts
               jours?\newline
               -Oui.\newline
               -Je le veux bien.\newline
               -Quand?\newline
               -Tout de suite. Seulement, je vous préviens que je le ferai à vos frais.\newline
               -C'est de la folie! s'écria Andrew
               Stuart\index[persName]{{Andrew Stuart}}, qui commençait à se vexer de l'insistance de son partenaire. Tenez! jouons,
               plutôt.\newline
               -Refaites alors, répondit Phileas Fogg\index[persName]{{Phileas Fogg}}, 
               car il y a « mal donne. »\newline
            
               Andrew Stuart\index[persName]{{Andrew Stuart}} reprît les cartes d'une
               main fébrile, puis, tout à coup, les posant
               sur la table :
               «Eh bien, oui, monsieur Fogg\index[persName]{{Phileas Fogg}}, dit-il,
               oui, je parie quatre mille livres! \newline
               -Mon cher Stuart\index[persName]{{Andrew Stuart}}, dit Fallentin\index[persName]{{Samuel Fallentin}}, calmez-vous. Ce n'est pas sérieux.\newline
               -Quand je dis je parie, répondit Andrew Stuart\index[persName]{{Andrew Stuart}}, c'est toujours sérieux.\newline
               -Soit » dit Mr. Fogg\index[persName]{{Phileas Fogg}}. Puis se tournant
               vers ses collègues:
                « J'ai vingt mille livres (500,000 fr.) déposes chez Baring frères\index[placeName]{{Baring frères}}. Je les risquerai
               volontiers.\newline
               -Vingt mille livres ! s'écria John Sulivan. Vingt mille livres qu'un retard im
               
                  
               prévu peut vous faire perdre !\newline
               -L'imprévu n'existe pas, répondit simplement Phileas Fogg\index[persName]{{Phileas Fogg}}.\newline
               -Mais, monsieur Fogg\index[persName]{{Phileas Fogg}}, ce laps de quatre-vingts jours n'est calculé que comme
               un minimum de temps!\newline
               -Un minimum bien employé suffit à 
               tout.\newline
               -Mais pour ne pas le dépasser, il faut
               sauter mathématiquement des railways 
               dans les paquebots, et des paquebots dans
               les chemins de fer!\newline
               -Je sauterai mathématiquement.\newline
               -C'est une plaisanterie !\newline
               -Un bon Anglais ne plaisante jamais
               quand il s’agit d'une chose aussi sérieuse
               qu'un pari, répondit Phileas Fogg\index[persName]{{Phileas Fogg}}. Je parie vingt mille livres contre qui voudra,
               que je ferai le tour de la terre en quatrevingts jours ou moins, soit dix-neuf cent 
               vingt heures ou cent quinze mille deux
               cents minutes. Acceptez-vous?\newline
               -Nous acceptons, répondirent MM. 
               Stuart\index[persName]{{Andrew Stuart}}, Fallentin\index[persName]{{Samuel Fallentin}}, Sullivan\index[persName]{{John Sullivan}}, Flanagan\index[persName]{{Thomas Flanagan}} et
               Ralph\index[persName]{{Gauthier Ralph}}\newline, après s'être entendus.
               -Bien, dit Fogg\index[persName]{{Phileas Fogg}}. Le train de Douvres\index[placeName]{{Douvres}}
               part à dix heures trente-cinq. Je le prendrai.\newline
               -Ce soir même ? demanda Stuart\index[persName]{{Andrew Stuart}}.\newline
               -Ce soir même, répondit Mr. Fogg\index[persName]{{Phileas Fogg}}. \newline
               -Donc, ajouta-t-il, en consultant un calendrier de poche puisque c'est aujourd'hui mercredi 2 octobre, je devrai être
               de retour à Londres\index[placeName]{{Londres}}, dans ce salon même 
               du Reform-Club\index[placeName]{{Reform-Club}}, le samedi 21 décembre, à
               dix heures trente-cinq du soir, faute de
               quoi les vingt mille livres déposées actuellement à mon crédit chez Baring frères\index[placeName]{{Baring frères}}
               vous appartiendront de fait et de droit,
               messieurs. Voici un chèque de pareille 
               somme. »\newline
            
               Un procès-verbal du pari fut fait, et signé sur-le-champ par les six cointéressés.
               Phileas Fogg\index[persName]{{Phileas Fogg}} était demeuré froid. I1 n'avait certainement pas parié pour gagner,
               et n'avait engagé ces vingt mille livres, la moitié de sa fortune,-que parce qu'il 
               prévoyait qu'il pourrait avoir à dépenser
               l'autre pour mener à bien ce difficile, pour 
               
               ne pas dire inexécutable projet. Quant à
               ses adversaires, eux, ils paraissaient
               émus, non pas à cause de la valeur de
               l'enjeu, mais parce qu'ils se faisaient une
               sorte de scrupule de lutter contre l'impossible.
            
               Neuf heures sonnaient alors. On offrit à
                Mr. Fogg\index[persName]{{Phileas Fogg}} de suspendre le whist afin qu'il
               pût faire ses préparatifs de départ.
               « Je suis toujours prêt !» répondit cet 
               impassible gentleman, et donnant les cartes :\newline
               « Je retourne carreau, dit-il. A vous de
               jouer, monsieur Stuart\index[persName]{{Andrew Stuart}}. »\newline
            
            \chapter{Chapitre IV}
            
            
               A neuf heures vingt-cinq, Phileas Fogg\index[persName]{{Phileas Fogg}},
               après avoir gagné une vingtaine de guinées au whist, prit congé de ses honorables collègues, et quitta le Reform-Club\index[placeName]{{Reform-Club}}.
               A neuf heures quarante-cinq, il ouvrait la
               porte de sa maison et rentrait chez lui.
            
               Passepartout\index[persName]{{Jean Passepartout}}, qui avait consciencieuse
               ment étudié son programme, fut assez surpris en voyant Mr. Fogg\index[persName]{{Phileas Fogg}}, coupable d'inexactitude, apparaitre à cette heure insolite.
               Suivant la notice, le locataire de Savillerow\index[placeName]{{Saville-Row}} ne devait rentrer qu'à minuit précis.
            
               Phileas Fogg\index[persName]{{Phileas Fogg}} était tout d'abord monté 
               à sa chambre, puis il appela :
               « Passepartout\index[persName]{{Jean Passepartout}}. »\newline
               Passepartout\index[persName]{{Jean Passepartout}} ne répondit pas. Cet appel
               ne pouvait s'adresser à lui. Ce n'était pas 
               l'heure.
               « Passepartout\index[persName]{{Jean Passepartout}}, »\newline reprit Mr. Fogg\index[persName]{{Phileas Fogg}} sans 
               élever la voix davantage.
               Passepartout\index[persName]{{Jean Passepartout}} se montra.
               « C'est la deuxième fois que je vous 
               appelle, dit Mr. Fogg\index[persName]{{Phileas Fogg}}.\newline
               -Mais il n'est pas minuit, répondit
               Passepartout\index[persName]{{Jean Passepartout}}, sa notice à la main.\newline
               -Je le sais, reprit Phileas Fogg\index[persName]{{Phileas Fogg}}, et je
               ne vous fais pas de reproche. Nous partons dans dix minutes pour Douvres\index[placeName]{{Douvres}}
               et Calais\index[placeName]{{Calais}}. »\newline 
               
               Une sorte de grimace s'ébaucha sur la 
               ronde face du Français. Il était évident
               qu'il avait mal entendu.
               « Monsieur se déplace? demanda-t-il.\newline
               -Oui, répondit Phileas Fogg\index[persName]{{Phileas Fogg}}. Nous allons faire le tour du monde. »\newline
            
               Passepartout\index[persName]{{Jean Passepartout}}, l'œil démesurément ouvert, la paupière et le sourcil surélevés,
               les bras détendus, le corps affaissé, présentait alors tous les symptômes de l'étonnement poussé jusqu'à la stupeur.
            
               « Le tour du monde! murmura-t-il.\newline
               -En quatre-vingts jours. répondit M.
               Fogg\index[persName]{{Phileas Fogg}}. Ainsi, nous n'avons pas un instant 
               à perdre.\newline
               -Mais les malles?. dit Passepartout\index[persName]{{Jean Passepartout}}, 
               qui balançait inconsciemment sa tête de 
               droite et de gauche.\newline
               -Pas de malles, un sac de nuit seulement. Dedans, deux chemises de laine,
               trois paires de bas. Autant pour vous. Nous 
               achèterons en route. Vous descendrez mon
               makintosh et ma couverture de voyage. 
               Ayez de bonnes chaussures. D'ailleurs,
               nous marcherons peu ou pas. Allez. » \newline
            
               Passepartout\index[persName]{{Jean Passepartout}} aurait voulu répondre. Il
               ne put. Il quitta la chambre de Mr. Fogg\index[persName]{{Phileas Fogg}},
               monta dans la sienne, tomba sur une 
               chaise, et employant une phrase assez 
               vulgaire de son pays :
            
               « Ah bien se dit-il, elle est forte, cellelà! Moi qui voulais rester tranquille!… »\newline
            
               Et machinalement, il fit ses préparatifs 
               de départ. Le tour du monde en quatrevingts jours ! Avait-il affaire à un fou? 
               Non. C'était une plaisanterie ? On allait à
               Douvres\index[placeName]{{Douvres}}, bien. A Calais\index[placeName]{{Calais}}, soit. Après tout, 
               cela ne pouvait notablement contrarier le
               brave garçon, qui depuis cinq ans n'avait 
               pas foulé le sol de la patrie. Peut-être même irait-on jusqu'à Paris\index[placeName]{{Paris}}, et, ma foi, il 
               reverrait avec plaisir la grande capitale.
               Mais, certainement, un gentleman aussi 
               ménager de ses pas, s'arrêterait là. Oui,
               sans doute, mais il n'en était pas moins
               vrai qu'il partait, qu'il se déplaçait, ce
               gentleman si casanier jusqu'alors!
            
               A dix heures, Passepartout\index[persName]{{Jean Passepartout}} avait préparé le modeste sac qui contenait sa garde- 
            
               robe et celle, de son maître, et, l'esprit
               encore troublé, il quitta sa chambre dont 
               il ferma soigneusement la porte. Il rejoignit Mr. Fogg\index[persName]{{Phileas Fogg}}.
            
               Mr. Fogg\index[persName]{{Phileas Fogg}} était prêt. Il portait sous son 
               bras le Bradshaw's continental railway,
                steam transit and general guide, qui devait 
               lui fournir toutes les indications nécessaires à son voyage. Il prit le sac des 
               mains de Passepartout\index[persName]{{Jean Passepartout}}, l'ouvrit et y glissa
               une forte liasse de ces belles bank-notes 
               qui ont cours dans tous les pays.
            
               « Vous n'avez rien oublié? demanda-t-il.\newline
               -Rien, monsieur.\newline
               -Mon makintosch et ma couverture?\newline
               -Les voici.\newline
               -Bien, prenez ce sac. »
               Mr. Fogg\index[persName]{{Phileas Fogg}} remit le sac à Passepartout\index[persName]{{Jean Passepartout}}.\newline
               « Et ayez-en soin, ajouta-il. Il y a vingt
               mille livres dedans (500,000 francs). »\newline 
            
               Le sac faillit s'échapper des mains de 
               Passepartout\index[persName]{{Jean Passepartout}}, comme si les vingt mille
               livres eussent été en or et pesé considérablement.
            
               Le maître et le domestique descendirent 
               alors, et la porte de la rue fut fermée à
               double tour.
            
               Une station de voitures se trouvait à
               l'extrémité de Saville-row\index[placeName]{{Saville-Row}}. Mr. Phileas
               Fogg\index[persName]{{Phileas Fogg}} et son domestique montèrent dans 
               un cab qui se dirigea rapidement vers la 
               gare de Charing-Cross\index[placeName]{{}}, à laquelle aboutit
               un des embranchements du South Eastern 
               railway.
            
               A dix heures treize, le cab s'arrêta devant la grille de la gare. Passepartout\index[persName]{{Jean Passepartout}} 
               sauta à terre. Son maître le suivit et paya 
               le cocher.
            
               En ce moment, une pauvre mendiante,
               tenant un enfant à la main, pieds nus 
               dans la boue, coiffée d'un chapeau dépenaillé auquel pendait une plume lamentable, un châle en loques sur ses haillons,
               s'approcha de Mr. Phileas Fogg\index[persName]{{Phileas Fogg}} et lui demanda l'aumône.
            
               Phileas Fogg\index[persName]{{Phileas Fogg}} tira de sa poche les vingt
               guinées qu'il venait de gagner au whist, 
               et les présentant à la mendiante :
            
               « Tenez, ma brave femme, dit-il, je suis
                  
               
                  
               content de vous avoir rencontrée ! » \newline
            
               Puis, il passa.
            
               Passepartout\index[persName]{{Jean Passepartout}} eut comme une sensation
               d'humidité autour de la prunelle. Son
               maître avait fait un pas dans son cœur.
            
                Mr. Fogg\index[persName]{{Phileas Fogg}} et lui entrèrent aussitôt dans la
               grande salle de la gare. Là, Phileas Fogg\index[persName]{{Phileas Fogg}} 
               donna à Passepartout\index[persName]{{Jean Passepartout}} l'ordre de prendre 
               deux billets de première classe pour Paris. Puis, se retournant, il aperçut ses cinq 
               collègues du Reform-Club\index[placeName]{{Reform-Club}}.
            
               « Messieurs, je pars, dit-il, et les divers
               visas apposés sur un passe-port que j'emporte à cet effet, vous permettront, au retour, de contrôler mon itinéraire.\newline
               -Oh! Mr. Fogg\index[persName]{{Phileas Fogg}}, répondit poliment Gauthier Ralph\index[persName]{{Gauthier Ralph}}, c'était inutile. Nous nous en
               rapportons à votre honneur de gentleman !\newline
               -Cela vaut mieux ainsi, dit Mr. Fogg\index[persName]{{Phileas Fogg}} \newline
               -Vous n'oubliez pas que vous devez
               être revenu ? fit observer Andrew 
               Stuart\index[persName]{{Andrew Stuart}}…\newline
               -Dans quatre-vingts jours, répondit 
               Mr. Fogg\index[persName]{{Phileas Fogg}}, le samedi, 21 décembre 1872, à 
               dix heures trente-cinq minutes du soir. Au
               revoir, messieurs. »\newline
            
               A dix heures et demie, Phileas Fogg\index[persName]{{Phileas Fogg}} et
               son domestique prirent place dans le 
               même compartiment. A dix heures trentecinq, un coup de sifflet retentit, et le train
               se mit en marche.
            
               La nuit était noire. Il tombait une pluie
               fine. Phileas Fogg\index[persName]{{Phileas Fogg}}, accoté dans son coin,
               ne parlait pas. Passepartout\index[persName]{{Jean Passepartout}}, encore abasourdi, pressait machinalement contre lui 
               le sac aux bank-notes.
            
               Mais le train n'avait pas dépassé Sydenham\index[placeName]{{Sydenham}}, que Passepartout\index[persName]{{Jean Passepartout}} poussait un véritable cri de désespoir!
            
               « Qu'avez-vous ? demanda Mr. Fogg\index[persName]{{Phileas Fogg}}.\newline
               -Il y a… que… dans ma précipitation…
               mon trouble… j'ai oublié…\newline
               -Quoi?\newline
               -D'éteindre le bec de gaz de ma chambre !\newline
               -Eh bien, mon garçon, répondit froidement Mr.Fogg\index[persName]{{Phileas Fogg}}, il brûle à votre compte !»\newline 
               JULES VERNE.
               (A suivre.)
            
            
            \printindex[persName]
            \addcontentsline{toc}{chapter}{Index des noms de personnages}
            \printindex[placeName]
            \addcontentsline{toc}{chapter}{Index des noms de lieux}
            
            \tableofcontents
            \addcontentsline{toc}{chapter}{Table des matières}
            
            \end{document}
        